\begin{answer}
    \rpos 
    \begin{answerenum}
        \item The $\sigma$-algebra $\mathcal{G}$ generated by the partition $\{A_1, \ldots, A_n\}$ consists of all possible unions of the sets in the partition. Since the sets $A_i$ are disjoint and cover the entire sample space $\Omega$, any event in $\mathcal{G}$ can be expressed as a union of some subset of the $A_i$. Therefore, we have:
            \[
            \mathcal{G} = \left\{ \bigcup_{j \in J} A_j : J \subseteq \{1, 2, \ldots, n\} \right\}.
            \]
            This includes the empty set (when $J = \emptyset$) and the entire space $\Omega$ (when $J = \{1, 2, \ldots, n\}$).

            This can be shown rigorously by using the double inclusion. The easy inclusion is that any union of the sets $A_j$ is in $\mathcal{G}$ by definition. 

            Now it is left to show that \( \mathcal{G} \subseteq \{ \bigcup_{j \in J} A_j : J \subseteq \{1, 2, \ldots, n\} \} \).
            The trick here is to \textbf{show that \( \{ \bigcup_{j \in J} A_j : J \subseteq \{1, 2, \ldots, n\} \} \) is a $\sigma$-algebra.} Then the inclusion follows from the definition of $\mathcal{G}$ as the smallest $\sigma$-algebra containing the sets \( A_1, \ldots, A_n \).

        \item To show that 
            \[
            \mathbb{E}(X \mid \mathcal{G})(\omega) = \sum_{j:\,\mathbb{P}(A_j) > 0} \frac{\mathbb{E}(X1_{A_j})}{\mathbb{P}(A_j)} 1_{A_j}(\omega),
            \]
            we start by noting that $\mathbb{E}(X \mid \mathcal{G})$ is $\mathcal{G}$-measurable. This means that it is constant on each set $A_j$ of the partition. Therefore, for each $j$, there exists a constant $c_j$ such that:
            \[
            \mathbb{E}(X \mid \mathcal{G})(\omega) = c_j, \quad \text{for } \omega \in A_j.
            \]
            To find $c_j$, we use the property of conditional expectation:
            \[
            c_j = \mathbb{E}(X \mid A_j) = \frac{\mathbb{E}(X 1_{A_j})}{\mathbb{P}(A_j)}.
            \]
            Thus, we can express $\mathbb{E}(X \mid \mathcal{G})(\omega)$ as:
            \[
            \mathbb{E}(X \mid \mathcal{G})(\omega) = \sum_{j=1}^n c_j 1_{A_j}(\omega) = \sum_{j:\,\mathbb{P}(A_j) > 0} \frac{\mathbb{E}(X1_{A_j})}{\mathbb{P}(A_j)} 1_{A_j}(\omega).
            \]
            This completes the proof.
    \end{answerenum}
\end{answer}