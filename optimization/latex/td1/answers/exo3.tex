\begin{answer}
    \rpos
    \begin{answerenum}
        \item The critical points of \(f_1\) can be found by computing the gradient and setting it to zero:
            \[
            \nabla f_1(x,y) = \begin{pmatrix}
            2(x-y) + 3(x+y)^2 \\
            2(y-x) + 3(x+y)^2
            \end{pmatrix} = 0.
            \]
            This gives us a system of equations to solve for the critical points.
            Simplifying the equations,
            \[
            \begin{cases}
                3(x+y)^2 = 2(x-y) \\
                3(x+y)^2 = -2(x-y)
            \end{cases}
            \]
            Which leads to \(x = y = 0\) as the only critical point.
            This critical point is a saddle point, and we only need to take the direction \(\vec{d}=(1, 1)\).
        \item For \(f_2\), it is worth noticing that \(f_2\) can be expressed with a quadratic form:
            \[
            f_2(x, y) = \frac{1}{2} \begin{pmatrix} x \\ y \end{pmatrix}^T \begin{pmatrix} 2 & 3 \\ 3 & -4 \end{pmatrix} \begin{pmatrix} x \\ y \end{pmatrix}
            \]
            we compute the gradient:
            \[
            \nabla f_2(x,y) = \begin{pmatrix} 2 & 3 \\ 3 & -4 \end{pmatrix} \begin{pmatrix} x \\ y \end{pmatrix} 
            \]
            Again, we have a system of equations to solve. The unique solution is given by:
            \[
            \begin{pmatrix} 2 & 3 \\ 3 & -4 \end{pmatrix} \begin{pmatrix} x \\ y \end{pmatrix} = 0
            \]
            which leads to \(x = 0\) and \(y = 0\).
            To classify this critical point, we compute the Hessian matrix:
            \[
            \nabla^2 f_2(x,y) = \begin{pmatrix}
            2 & 3 \\
            3 & -4
            \end{pmatrix}
            \]
            The eigenvalues of this Hessian matrix can be found by solving the characteristic polynomial:
            \[
            \det\left(\nabla^2 f_2(x,y) - \lambda I\right) = 0
            \]
            which simplifies to:
            \[
            \det\begin{pmatrix}
            2 - \lambda & 3 \\
            3 & -4 - \lambda
            \end{pmatrix} = 0
            \]
            The characteristic polynomial is given by:
            \[
            (2 - \lambda)(-4 - \lambda) - 9 = 0
            \]
            which leads to:
            \[
            \lambda^2 + 2\lambda - 17 = 0
            \]
            The eigenvalues are:
            \[
            \lambda_{1,2} = -1 \pm \sqrt{18} = -1 \pm 3\sqrt{2}
            \]
            Since one eigenvalue is positive and the other is negative, the critical point is a saddle point.

        \item Finally, for \(f_3\):
            \[
            \nabla f_3(x,y) = \begin{pmatrix}
            4x^3 \\
            3y^2 - 3
            \end{pmatrix} = 0.
            \]
            We can solve these equations to find the critical points. The solutions are:
            \[
            x = 0, \quad y = \pm 1
            \]
            To classify these critical points, we compute the Hessian matrix:
            \[
            \nabla^2 f_3(x,y) = \begin{pmatrix}
            12x^2 & 0 \\
            0 & 6y
            \end{pmatrix}
            \]
            The eigenvalues of this Hessian matrix are given by the diagonal elements:
            \[
            \lambda_1 = 12x^2, \quad \lambda_2 = 6y
            \]
            At both critical points the Hessian is degenerate in the $x$-direction (entry $12x^2=0$), so the usual second derivative test is inconclusive and we use higher-order expansion.

            1. Point $(0,1)$. Write $x=x-0, \; y=y-1$ (move first critical point to \(O\)):
            \[
            f_3(x,y)= x^4 + y^3 - 3y^2
            \]
            We examine $f(x,y) - f(0,0) = x^4 + y^3 - 3y^2$ along various paths through the origin:
            \begin{enumerate}
            \item \textbf{Along the $x$-axis} $(y = 0)$:
            \[
            f(x,0) = x^4 \geq 0
            \]
            with equality only at $x = 0$.

            \item \textbf{Along the $y$-axis} $(x = 0)$:
            \[
            f(0,y) = y^3 - 3y^2 = y^2(y - 3)
            \]
            For small $|y| > 0$: since $y - 3 < 0$, we have $f(0,y) < 0$.

            \item \textbf{Along the parabola} $y = x^2$:
            \[
            f(x,x^2) = x^4 + x^6 - 3x^4 = x^6 - 2x^4 = x^4(x^2 - 2)
            \]
            For small $|x| > 0$: since $x^2 - 2 < 0$, we have $f(x,x^2) < 0$.
            \end{enumerate}
            Since the function takes both positive values (along the $x$-axis) and negative values (along the $y$-axis and the parabola $y = x^2$) in every neighborhood of $(0,0)$, we conclude that:
            \begin{center}
            \boxed{\text{$(0,0)$ is a \textbf{saddle point} of } f_3(x,y) = x^4 + y^3 - 3y^2}
            \end{center}
            
            2. Point $(0,-1)$. Write $y=-1+s$:
            \[
            f_3(x,-1+s)= x^4 + (-1+s)^3 -3(-1+s) -2 = x^4 -3 s^2 + s^3.
            \]
            Thus
            \[
            f_3(x,-1+s)-f_3(0,-1)= x^4 -3 s^2 + s^3.
            \]
            Along $s=0$, $x\neq 0$: difference $=x^4>0$. Along $x=0$, $0< s <3$: difference $= -3s^2 + s^3 = -3s^2(1 - s/3)<0$. Hence values of both signs occur arbitrarily close to $(0,-1)$: $(0,-1)$ is a saddle point.

            \textbf{Conclusion:}
            \begin{center}
            \boxed{\text{Both $(0,1)$ and $(0,-1)$ are \textbf{saddle points} of } f_3(x,y) = x^4 + y^3 - 3y -2 }
            \end{center}

    \end{answerenum}
\end{answer}