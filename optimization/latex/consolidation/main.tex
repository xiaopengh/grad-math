\documentclass[a4paper, 12pt]{article}
\input{../../../general-preamble.tex}
\usepackage{listings}
\usepackage{xcolor}

\lstset{
    language=R,
    basicstyle=\ttfamily\small,
    keywordstyle=\color{blue},
    commentstyle=\color{gray},
    stringstyle=\color{red},
    numbers=left,
    numberstyle=\tiny,
    frame=single,
    breaklines=true
}


\begin{document}

{   \centering
    \Large\textbf{Optimization Midterm 2025 Autumn}\\[0.5em]
    \today\\[1em] 
}

\section{Optimality Conditions}

\begin{answerenum}
    \item A useful property for symmetric definite positive matrices: 
        Let \(A \in S_d^{++}\) then we have the following inequality for all \(z \in \mathbb{R}^d\):
        \[ \lambda_{\min}(A) \|z\|^2 \leq \langle Az, z \rangle \leq \lambda_{\max}(A) \|z\|^2 \]
        \textbf{Remark:} This is very useful when dealing with quadratic forms. Also note that this is very useful to prove the second order conditions for optimality. (i.e. if the Hessian \(\nabla^2 f(x)\) is positive definite at a critical point \(x^*\), then \(x^*\) is a local minimum.)
    \item \textbf{First Order Optimality Condition:} 
        Let \(f: \mathbb{R}^d \to \mathbb{R}\) be a differentiable function. A necessary condition for \(x^* \in \mathbb{R}^d\) to be a local minimum of \(f\) is that the gradient at that point is zero:
        \[ \nabla f(x^*) = 0 \]
    \item \textbf{Second Order Optimality Condition:}
        Let \(f: \mathbb{R}^d \to \mathbb{R}\) be a twice differentiable function. Let \(x^*\) be a critical point of \(f\) (i.e. \(\nabla f(x^*) = 0\)). Then:
        \begin{itemize}
            \item If the Hessian \(\nabla^2 f(x^*)\) is positive definite, then \(x^*\) is a local minimum of \(f\).
            \item If the Hessian \(\nabla^2 f(x^*)\) has 2 eigenvalues of opposite signs, then \(x^*\) is a saddle point of \(f\).
            \item If the Hessian \(\nabla^2 f(x^*) \in S_d^{+} \) but is not positive definite, then \(x^*\) could be a local minimum or a saddle point (inconclusive).
        \end{itemize} 
\end{answerenum}

\remark{\textbf{Some useful multicalculus rules:} 
    \begin{itemize}
        \item \textbf{Product rule:} If \(u, v: \mathbb{R}^d \to \mathbb{R}^d\) are differentiable functions, then:
            \[ D_x \, \langle u, v \rangle = \langle D_x \, u , v \rangle + \langle u , D_x \, v \rangle \] 
        \item \textbf{Chain rule:} If \(f: \mathbb{R}^m \to \mathbb{R}\) is a differentiable function and \(g: \mathbb{R}^d \to \mathbb{R}^m\) is a differentiable function, then:
            \[ \underbrace{\frac{\partial f}{\partial x}}_{n \times 1} = \underbrace{\frac{\partial g}{\partial x}}_{n \times m} \cdot \underbrace{\frac{\partial f}{\partial g}}_{m \times 1} \]
    \end{itemize}
}

\newpage

\section{Convexity}
\begin{answerenum}
    \item \textbf{Tangent plane property of convex functions:} 
        Let \(f: \mathbb{R}^d \to \mathbb{R}\) be a differentiable convex (\(\alpha\) strongly convex) function. Then for all \(x, y \in \mathbb{R}^d\), we have:
        \[ f(y) \geq f(x) + \langle \nabla f(x), y - x \rangle \]
        \[ f(y) \geq f(x) + \langle \nabla f(x), y - x \rangle + \frac{\alpha}{2}\|y-x\|^2 \]
        This means that the function lies above its tangent plane (plus a quadratic term for strong convexity) at any point.
    
    \item \textbf{Monotonicity of the gradient}
        Let \(f: \mathbb{R}^d \to \mathbb{R}\) be a differentiable convex (\(\alpha\) strongly convex) function. Then for all \(x, y \in \mathbb{R}^d\), we have:
        \[ \langle \nabla f(y) - \nabla f(x), y - x \rangle \geq 0 \]
        \[ \langle \nabla f(y) - \nabla f(x), y - x \rangle \geq \alpha \|y - x\|^2 \]
        This means that the gradient of a convex function is a monotone operator (stronger monotonicity than a quadratic function for strongly convex functions).
    
    \item \textbf{Characterization of convexity via Hessian:} 
        Let \(f: \mathbb{R}^d \to \mathbb{R}\) be a twice differentiable function. Then:
        \begin{itemize}
            \item \(f\) is convex \colorbox{yellow}{if and only if} its Hessian is positive semi-definite for all \(x \in \mathbb{R}^d\):
                \[ \nabla^2 f(x) \in S_d^{+}, \quad \forall x \in \mathbb{R}^d \]
            \item \(f\) is \(\alpha\)-strongly convex \colorbox{yellow}{if and only if} its Hessian's smallest eigenvalue is bounded below by \(\alpha\) for all \(x \in \mathbb{R}^d\):
                \[ \lambda_d(\nabla^2 f(x)) \geq \alpha, \quad \forall x \in \mathbb{R}^d \]
            \item If the Hessian is positive definite for all \(x \in \mathbb{R}^d\), then \(f\) is strictly convex, but the converse is not true. (e.g. \(f(x) = x^4\) is strictly convex but its Hessian is zero at \(x=0\).)
        \end{itemize}
    
    \item \textbf{Segementation function:} 
        Let \(f: \mathbb{R}^d \to \mathbb{R}\) be a differentiable function. Define the segmentation function \(\phi: [0,1] \to \mathbb{R}\) as:
        \[ \phi(t) = f(x + t(y - x)), \quad t \in [0,1] \]
        \[ \phi'(t) = (y - x)^T \cdot \nabla f(x + t(y - x)) = \langle \nabla f(x + t(y - x)), y - x \rangle \]
        for fixed \(x, y \in \mathbb{R}^d\). Then:
        \[ f(y) - f(x) = \int_0^1 \phi'(t) \, dt = \int_0^1 \langle \nabla f\left(x + t(y - x)\right), y - x \rangle \, dt \]
        This is useful for proving various properties: 
        \begin{itemize}
            \item If we want to prove the tangent plane property from the monotonicity of the gradient: 
                \begin{align*}
                    f(y) - f(x) &= \int_0^1 \langle \nabla f\left(x + t(y - x)\right), y - x \rangle \, dt \\
                    &= \int_0^1 \langle \nabla f\left(x + t(y - x)\right) - \nabla f(x) + \nabla f(x), y - x \rangle \, dt \\
                    &= \int_0^1 \langle \nabla f\left(x + t(y - x)\right) - \nabla f(x), y - x \rangle \, dt + \langle \nabla f(x), y - x \rangle \\
                    &\geq 0 + \langle \nabla f(x), y - x \rangle \quad \text{(by monotonicity of the gradient)}
                \end{align*}
            \item If we want to prove that \(f\) is convex from \(\nabla^2 f(x) \in S_d^{+}\) for all \(x\):
                \begin{align*}
                    f(y) - f(x) &= \int_0^1 \langle \nabla f\left(x + t(y - x)\right), y - x \rangle \, dt \\
                    &= \int_0^1 \langle \nabla f(x + t(y - x)) - \nabla f(x) + \nabla f(x), y - x \rangle \, dt 
                \end{align*}
                Similarly as before, we can define \( \varphi(t) = \langle \nabla f (x + t(y - x)), y - x \rangle \)
                \[ \varphi'(t) = (y - x)^T \cdot \nabla^2 f(x + t(y - x)) \cdot (y - x) \geq 0 = \langle \nabla^2 f(x + t(y - x))(y - x), y - x \rangle \]
                \[ f(y) - f(x) = \int_0^1 \int_0^t \varphi'(s) \, ds \, dt + \langle \nabla f(x), y - x \rangle \]
                Since \( \nabla^2 f \) is positive semi-definite, we have \( \varphi'(s) \geq 0\), which implies that \( f(y) - f(x) \geq \langle \nabla f(x), y - x \rangle \).
        \end{itemize}

\end{answerenum}

\newpage

\subsection{General Convexity Exercises}

\reminder{\textbf{Reminder}: A function \(f: \mathbb{R}^d \to \mathbb{R}\) is convex if for all \(x, y \in \mathbb{R}^d\) and \(t \in [0,1]\), we have
\[f(tx + (1-t)y) \leq t f(x) + (1-t) f(y).\]

A function \(f\) is strongly convex with parameter \(\alpha > 0\) if for all \(x, y \in \mathbb{R}^d\) and \(t \in [0,1]\), we have
\[f(tx + (1-t)y) \leq t f(x) + (1-t) f(y) - \frac{\alpha}{2} t(1-t) \|x - y\|^2.\]}

\begin{exercise}
    Let \(f \in \mathcal{C}^1 (\mathbb{R}^d, \mathbb{R}) \) be convex. Is it true that \(f\) has a unique critical point? 
\end{exercise}

It is not necessarily true that a convex function has a unique critical point. As a counterexample, consider the convex function \(f: \mathbb{R} \to \mathbb{R}\) defined by
\[f(x) = e^x.\] 
This function is convex since its second derivative \(f''(x) = e^x > 0\) for all \(x \in \mathbb{R}\). However, it has no critical points because its first derivative \(f'(x) = e^x\) is never zero. \qed

\begin{exercise}
    Let \(f \in \mathcal{C}^1 (\mathbb{R}^d, \mathbb{R}) \) be convex. Show that if \(f\) has a critical point at \(x^* \in \mathbb{R}^d\), then \(x^*\) is a global minimum of \(f\).
\end{exercise}

If \(\nabla f(x^*) = 0\) for some \(x^* \in \mathbb{R}^d\), then for any \(x \in \mathbb{R}^d\), the tangent plane property of convex functions gives us
\[f(x) \geq f(x^*) + \langle \nabla f(x^*), x - x^* \rangle = f(x^*) + 0 = f(x^*).\]

Thus, \(f(x) \geq f(x^*)\) for all \(x \in \mathbb{R}^d\), which means that \(x^*\) is a global minimum of \(f\). \qed

\begin{exercise}
    Let \(f \in \mathcal{C}^1 (\mathbb{R}^d, \mathbb{R}) \) be strictly convex. Show that \(f\) has at most one critical point.
\end{exercise}

Assume for the sake of contradiction that \(f\) has two distinct critical points \(x_1^*\) and \(x_2^*\) in \(\mathbb{R}^d\). By the definition of critical points, we have \(\nabla f(x_1^*) = 0\) and \(\nabla f(x_2^*) = 0\). Since \(f\) is strictly convex, for any \(\lambda \in (0,1)\), we have
\[f(\lambda x_1^* + (1-\lambda) x_2^*) < \lambda f(x_1^*) + (1-\lambda) f(x_2^*).\]
However, since both \(x_1^*\) and \(x_2^*\) are critical points, we have
\[f(x_1^*) = f(x_2^*).\]
Thus,
\[f(\lambda x_1^* + (1-\lambda) x_2^*) < f(x_1^*),\]
which contradicts the fact that both \(x_1^*\) and \(x_2^*\) are minima. Therefore, \(f\) can have at most one critical point. \qed

\subsection{Strong Convexcity Exercises}

\begin{exercise}
    Show that \(f\) is strongly convex with parameter \(\alpha > 0\), if and only if the function \(g(x) = f(x) - \frac{\alpha}{2} \|x\|^2\) is convex. 
\end{exercise}

The \( \implies \) direction: Assume that \(f\) is strongly convex with parameter \(\alpha > 0\). For any \(x, y \in \mathbb{R}^d\) and \(t \in [0,1]\), we have
\[f(tx + (1-t)y) \leq t f(x) + (1-t) f(y) - \frac{\alpha}{2} t(1-t) \|x - y\|^2.\]

Let's first deduce a magic formula to replace \(\|tx + (1-t)y\|^2\)

\[\begin{aligned}
\|tx + (1-t)y\|^2
&= \langle tx + (1-t)y,\; tx + (1-t)y\rangle \\
&= t^2\|x\|^2 + (1-t)^2\|y\|^2 + 2t(1-t)\langle x, y\rangle.
\end{aligned}\]

\[\|x-y\|^2 = \|x\|^2 + \|y\|^2 - 2\langle x,y\rangle
\quad\Longrightarrow\quad
\langle x,y\rangle = \frac{\|x\|^2 + \|y\|^2 - \|x-y\|^2}{2}\]

Plugging this into the \( \|tx + (1-t)y\|^2 \) expression, we get

\[\boxed{\|tx + (1-t)y\|^2
= t\|x\|^2 + (1-t)\|y\|^2 - t(1-t)\|x-y\|^2.}\]

Now for \(g(tx + (1-t)y)\) we have: 
\begin{align*}
    g(tx + (1-t)y) 
    &= f(tx + (1-t)y) - \frac{\alpha}{2} \|tx + (1-t)y\|^2 \\
    &= \underbrace{f(tx + (1-t)y)}_{\text{Strongly convex}} - \frac{\alpha}{2} \left( t\|x\|^2 + (1-t)\|y\|^2 - t(1-t)\|x-y\|^2 \right) \\
\end{align*}
\vspace{-3em}
\begin{align*}
    &\leq tf(x) + (1-t)f(y) - \frac{\alpha}{2} t(1-t) \|x - y\|^2 - \frac{\alpha}{2} \left( t\|x\|^2 + (1-t)\|y\|^2 - t(1-t)\|x-y\|^2 \right) \\
    &\leq tf(x) + (1-t)f(y) - \frac{\alpha}{2} \left( t\|x\|^2 + (1-t)\|y\|^2 \right) \\
    &\leq tg(x) + (1-t)g(y)
\end{align*}

The \(\impliedby\) direction follows by assuming \(g\) is convex and reversing the steps above by using the magic formula for \(\|tx + (1-t)y\|^2\).

\begin{exercise}[Some examples of strongly convex functions]
    Determine which of the following functions are strongly convex on \(\mathbb{R}^d\):
    \begin{enumerate}
        \item \(f(x) = \|x\|^2\)
        \item \(f(x) = e^{\|x\|}\)
        \item \(f(x) = \frac{1}{2} x^T A x + b^T x + c\), where \(A\) is a symmetric positive definite matrix.
    \end{enumerate}
\end{exercise}

Recall that a function \(f\) is strongly convex with parameter \(\alpha > 0\) if for all \(x, y \in \mathbb{R}^d\) and \(t \in [0,1]\), we have
\[f(tx + (1-t)y) \leq t f(x) + (1-t) f(y) - \frac{\alpha}{2} t(1-t) \|x - y\|^2.\]

\begin{enumerate}
    \item For \(f(x) = \|x\|^2\), it is strongly convex with parameter \(\alpha = 2\), because by the tangent plane property of strongly convex functions, we have
    \[ \| y \|^2 = \|x + y - x \|^2 = \| x \|^2 + 2 \langle x, y - x \rangle + \| y - x \|^2 \]
    This shows that it is at most strongly convex with parameter 2.
    \item The Hessian of \(f(x) = e^{\|x\|}\)
    Calculate its gradient and Hessian we have 
    \[ \nabla f(x) = e^{\|x\|}\frac{x}{\|x\|} \]
    \[ \nabla^2 f(x) = e^{\|x\|}\left(\frac{I}{\|x\|} + \frac{x x^\top}{\|x\|^2} - \frac{x x^\top}{\|x\|^3}\right) \]
    \[ 
    \text{Eigenvalues of } \nabla^2 f(x):
    \begin{cases}
    e^{\|x\|} & \text{(radial direction)} \\
    \frac{e^{\|x\|}}{\|x\|} & \text{(any of the \(n-1\) orthogonal directions)}
    \end{cases} 
    \]
    We conclude that the function is strongly convex on (\( x \neq 0 \)) with parameter 
    \[
    \alpha = 
    \begin{cases}
        e^{\|x\|} & \text{if } \|x\| \geq 1 \\
        \frac{e^{\|x\|}}{\|x\|} & \text{if } 0 < \|x\| < 1
    \end{cases} 
    \]
    \item For \(f(x) = \frac{1}{2} x^T A x + b^T x + c\), where \(A\) is a symmetric positive definite matrix, the Hessian of \(f\) is given by \(\nabla^2 f(x) = A\). Since \(A\) is positive definite, there exists a constant \(\alpha > 0\) such that for all \(x \in \mathbb{R}^d\),
    \[x^T A x \geq \alpha \|x\|^2.\]
    Therefore, \(f\) is strongly convex with parameter \(\alpha\). \qed
\end{enumerate}

\begin{exercise}
    Let \(f \in \mathcal{C}^2 (\mathbb{R}^d, \mathbb{R}) \) be \(\alpha\)-strongly convex. Does \(f\) admit a unique minimizer?
\end{exercise}

By the strong convexity of \(f\), we know that under the assumption of the existence of a minimizer, it must be unique. Now we show the coercivity of \(f\) to ensure the existence of a minimizer. 

One way to do this is to show that \(f(x) \to +\infty\) as \(\|x\| \to +\infty\). 
Since \(f\) is \(\alpha\)-strongly convex, by the tangent plane property, we have for any \(x, x_0 \in \mathbb{R}^d\):
\[f(x) \geq f(x_0) + \langle \nabla f(x_0), x - x_0 \rangle + \frac{\alpha}{2} \|x - x_0\|^2.\] 
For the sake of the clarity, let \(x_0 = 0\), we have
\[f(x) \geq f(0) + \langle \nabla f(0), x \rangle + \frac{\alpha}{2} \|x\|^2.\]
This is a quadratic function in \(\|x\|\) with a positive leading coefficient \(\frac{\alpha}{2}\), which implies that \(f(x) \to +\infty\) as \(\|x\| \to +\infty\). Therefore, \(f\) is coercive. \qed

A more rigorous way is to use the sublevel set definition: a function \(f\) is coercive if the sublevel sets \(\{x \in \mathbb{R}^d : f(x) \leq c\}\) are bounded for all \(c \in \mathbb{R}\). 
\[f(x) \;\geq\; f(0) + \langle \nabla f(0), x \rangle + \frac{\alpha}{2}\|x\|^2\]
Recall that Young's inequality states that for any \(a, b \geq 0\) and \(\epsilon > 0\), we have
\[ab \leq \frac{a^2}{2\epsilon} + \frac{\epsilon b^2}{2}.\]
Now we control the linear term by the quadratic term using Cauchy-Schwarz and Young's inequality (with \(\epsilon = \frac{\alpha}{2}\)):
\[\langle \nabla f(0), x \rangle \;\geq\; -\|\nabla f(0)\| \|x\| \;\geq\; -\frac{1}{\alpha} \|\nabla f(0)\|^2 - \frac{\alpha}{4} \|x\|^2\]
Plugging this back we have
\[f(x) \;\geq\; f(0) - \frac{1}{\alpha} \|\nabla f(0)\|^2 + \frac{\alpha}{4} \|x\|^2 \]
Thus the sublevel sets are bounded. Therefore, \(f\) is coercive. \qed



\newpage

\section{Lipschitz Continuity of the Gradient}
\begin{answerenum}
    \item \textbf{Definition:} 
        A differentiable function \(f: \mathbb{R}^d \to \mathbb{R}\) has a Lipschitz continuous gradient with constant \(\mu > 0\) if for all \(x, y \in \mathbb{R}^d\), we have:
        \[ \|\nabla f(y) - \nabla f(x)\| \leq \mu \|y - x\| \]

    \item \textbf{Several Implications for Lipschitz Continuity:}
        Let \(f: \mathbb{R}^d \to \mathbb{R}\) be a differentiable function with a Lipschitz continuous gradient with constant \(\mu > 0\). Then for all \(x, y \in \mathbb{R}^d\), we have:
        \begin{itemize}
            \item Upper bound on the Hessian eigenvalues:
                \[ \lambda_1(\nabla^2 f(x)) \leq \mu \]
            \item Quadratic upper bound using gradient at \(x\):
                \[ f(y) \leq f(x) + \langle \nabla f(x), y - x \rangle + \frac{\mu}{2} \|y - x\|^2 \]
        \end{itemize}
\end{answerenum}

\section{Caratheodory theorem}
\begin{answerenum}
    \item \textbf{Convex hull:} 
        The convex hull of a set \(\Omega \subset \mathbb{R}^d\) is defined as the smallest convex set containing \(\Omega\). 
        \[ \text{conv}(\Omega) := \bigcap \{ C \subset \mathbb{R}^d \mid C \text{ is convex and } \Omega \subset C \} \]
        It can be expressed as:
        \[ \text{conv}(\Omega) = \left\{ \sum_{i=1}^N \lambda_i x_i \mid N \in \mathbb{N}, x_i \in \Omega, \lambda_i \geq 0, \sum_{i=1}^N \lambda_i = 1 \right\} \]
    \item \textbf{Caratheodory theorem:} 
        Let \(\Omega \subset \mathbb{R}^d\). Then any point in the convex hull of \(\Omega\) can be expressed as a convex combination of at most \(d+1\) points from \(\Omega\). In other words, for any \(x \in \text{conv}(\Omega)\), there exist points \(x_1, x_2, \ldots, x_{d+1} \in \Omega\) and coefficients \(\lambda_1, \lambda_2, \ldots, \lambda_{d+1} \geq 0\) such that:
        \[ x = \sum_{i=1}^{d+1} \lambda_i x_i \quad \text{and} \quad \sum_{i=1}^{d+1} \lambda_i = 1 \]
\end{answerenum}

\section{Extreme Points and Krein-Milman Theorem}
\begin{answerenum}
    \item \textbf{Extreme Points:} 
        Let \(C\) be a convex set in a vector space. A point \(x \in C\) is called an extreme point of \(C\) if \textbf{it cannot be expressed as a convex combination of two distinct points in \(C\)}. Formally, \(x\) is an extreme point if whenever \(x = \lambda y + (1 - \lambda) z\) for some \(y, z \in C\) and \(\lambda \in (0, 1)\), it follows that \(y = z = x\).
    \item \textbf{Projection on closed convex sets:} 
        Let \(C \subset \mathbb{R}^d\) be a non-empty closed convex set. For any point \(x \in \mathbb{R}^d\), there exists a \textbf{unique} point \(P_C(x) \in C\) such that:
        \[ \|x - P_C(x)\| = \min_{y \in C} \|x - y\| \]
        The point \(P_C(x)\) is called the projection of \(x\) onto the set \(C\).
        The projection operator \(P_C: \mathbb{R}^d \to C\) is \textbf{1-Lipschitz} (non-expansive), meaning that for all \(x, y \in \mathbb{R}^d\):
        \[ \|P_C(x) - P_C(y)\| \leq \|x - y\| \]
    \item \textbf{Krein-Milman Theorem:} 
        Let \(C\) be a non-empty compact convex subset of a locally convex topological vector space. Then \(C\) is the closed convex hull of its extreme points. In other words, if we denote the set of extreme points of \(C\) by \(\text{ext}(C)\), then:
        \[ C = \overline{\text{conv}}(\text{ext}(C)) \] 
        where \(\overline{\text{conv}}(\text{ext}(C))\) denotes the closure of the convex hull of the extreme points of \(C\).

        In other words, every compact convex set can be "reconstructed" from its extreme points by taking all possible convex combinations of those points and then taking the closure of that set.
\end{answerenum}

\section{Polyak-Lojasiewicz Inequality}
\begin{answerenum}
    \item Let's remind first the definition and some properties of strongly convex functions.
        A differentiable function \(f: \mathbb{R}^d \to \mathbb{R}\) is called \(\alpha\)-strongly convex if for all \(x, y \in \mathbb{R}^d\):
        \[ f(y) \geq f(x) + \langle \nabla f(x), y - x \rangle + \frac{\alpha}{2} \|y - x\|^2 \quad \text{\textbf{(Function is above the Tangent Cone)}}\]
        \[ \langle \nabla f(y) - \nabla f(x), y - x \rangle \geq \alpha \|y - x\|^2 \quad \text{\textbf{(Curvature Condition)}}\]
    \item \textbf{Polyak-Lojasiewicz Inequality:}
        A differentiable function \(f: \mathbb{R}^d \to \mathbb{R}\) is said to satisfy the Polyak-Lojasiewicz (PL) inequality with constant \(\mu > 0\) if for all \(x \in \mathbb{R}^d\):
        \[ \frac{1}{2} \|\nabla f(x)\|^2 \geq \mu (f(x) - f^*) \]
        where \(f^* = \inf_{x \in \mathbb{R}^d} f(x)\) is the global minimum value of \(f\).
    \item We can show that \(\alpha\)-strongly convex functions satisfy the PL inequality with constant \(\mu = \alpha\). 
        Let \(f\) be an \(\alpha\)-strongly convex function and let \(x^*\) be its unique minimizer. Then for any \(x \in \mathbb{R}^d\):
        \begin{align*}
            f(x) &\geq f(x^*) + \langle \nabla f(x^*), x - x^* \rangle + \frac{\alpha}{2} \|x - x^*\|^2 \\
            &\geq f(x^*) + 0 + \frac{\alpha}{2} \|x - x^*\|^2 \quad \text{(since } \nabla f(x^*) = 0\text{)} \\
            &\Rightarrow f(x) - f(x^*) \geq \frac{\alpha}{2} \|x - x^*\|^2
        \end{align*}
        On the other hand, making a minor modification to the first property of strongly convex functions, we have:
        \begin{align*}
            f(x) - f(y) & \leq \langle \nabla f(x), x - y \rangle - \frac{\alpha}{2} \|x - y\|^2 \\
            & \leq \|\nabla f(x)\| \|x - y\| - \frac{\alpha}{2} \|x - y\|^2 \quad \text{(by Cauchy-Schwarz inequality)} \\
            & \leq \frac{\|\nabla f(x)\|^2}{2\alpha} \quad \text{(by maximizing the right-hand side over } \|x - y\| \text{)}
        \end{align*}
        Combining the two inequalities, we get:
        \[ f(x) - f(x^*) \leq \frac{\|\nabla f(x)\|^2}{2\alpha} \]
        Rearranging this gives the PL inequality:
        \[ \frac{1}{2} \|\nabla f(x)\|^2 \geq \alpha (f(x) - f(x^*)) \]
\end{answerenum}

\end{document}