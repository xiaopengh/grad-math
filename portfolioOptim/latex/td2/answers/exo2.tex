\begin{exercise}
Consider a portfolio $\Pi = (\pi_0, \pi)' \in \mathbb{R}^{d+1}$, $d \geq 1$.  

1) Give the expression of the terminal wealth $W_T^{\Pi}(x_0)$ at time $T$ for the buy and hold strategy $\Pi$ and the initial capital $x_0$.  

We consider now the optimisation problem
\[
(P_\sigma) \quad 
\begin{cases}
\sup_{\pi \in \mathbb{R}^d} \mathbb{E}(W_T^{\Pi}(x_0)) \\
\mathrm{var}(W_T^{\Pi}(x_0)) = x_0^2\sigma^2
\end{cases}
\]
where $\sigma > 0$ is fixed.  

2) Show that $(P_\sigma)$ is equivalent to
\[
\begin{cases}
\sup_{\pi \in \mathbb{R}^d} \mathbb{E}(W_T^{\Pi}(x_0)) \\
\mathrm{var}(W_T^{\Pi}(x_0)) \leq x_0^2\sigma^2
\end{cases}
\]

3) For $\lambda > 0$, we define the Lagrangian 
\[
L_\lambda(\pi) := \mathbb{E}[R_\pi] - \lambda(\mathrm{Var}(R_\pi) - \sigma^2).
\]

(a) Show that $\sup_{\pi \in \mathbb{R}^d} L_\lambda(\pi) < +\infty$ and that the supremum is attained for a particular value of $\pi$ that we note $\pi_\lambda$.  

(b) Can you choose $\lambda(\sigma) > 0$ such that $\mathrm{Var}(R_{\pi_{\lambda(\sigma)}}) = \sigma^2$?  

(c) Explain why the efficient frontier is $\{(\sigma(\pi_{\lambda(\sigma)}), \mathbb{E}(R_{\pi_{\lambda(\sigma)}})), \, \sigma > 0\}$.
\end{exercise}

\begin{answer}
    \rpos
    \begin{answerenum}
        \item The terminal wealth \( W_T^{\Pi}(x_0) \) at time \( T \) for the buy and hold strategy \( \Pi = (\pi_0, \pi)' \) with initial capital \( x_0 \) can be expressed as:
            \[
            W_T^{\Pi}(x_0) = x_0 \pi_0 R^0 + x_0 \pi' R_T
            \]
        \item Calculation the variance, we have
            \[
            \mathrm{var}(W_T^{\Pi}(x_0)) = x_0^2 \mathrm{var}(\pi' R_T) = x_0^2 \pi' \Sigma \pi
            \]
            \( \pi' \Sigma \pi \) is a quadratic form and \( \Sigma \) is positive definite. 
            We have \( \pi' \Sigma \pi \geq 0 \) and the equality holds if and only if \( \pi = 0 \). Thus, we have that the constraint set \( \{\pi \in \mathbb{R}^d : \mathrm{var}(W_T^{\Pi}(x_0)) = x_0^2\sigma^2\} = \{\pi \in \mathbb{R}^d : \pi' \Sigma \pi = \sigma^2\} \) is a compact set (closed and bounded ellipsoid in \( \mathbb{R}^d \)).

            Since the objective function \( \mathbb{E}(W_T^{\Pi}(x_0)) \) is continuous and linear in \( \pi \), and we are maximizing over a compact set, the maximum is attained by the \textbf{Extreme Value Theorem}.

            Moreover, since the constraint set is the boundary of the ellipsoid \( \{\pi : \pi' \Sigma \pi \leq \sigma^2\} \), any point in the interior would give a strictly smaller variance. By the scaling argument (if \( \pi' \Sigma \pi < \sigma^2 \), then \( k\pi \) with \( k > 1 \) appropriately chosen gives \( (k\pi)' \Sigma (k\pi) = \sigma^2 \) and higher expected return), the maximum over the inequality constraint is achieved on the boundary, i.e., where the equality constraint holds.
    \end{answerenum}
\end{answer}