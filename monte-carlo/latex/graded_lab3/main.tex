\documentclass[a4paper, 10pt]{article}
\setlength{\parindent}{0pt} % Disable the indent for new paragraphs

% Aesthetic packages
\usepackage[T1]{fontenc} % select T1 font encoding: suitable for Western European Latin scripts
\usepackage[most]{tcolorbox}  % for styled boxes
\usepackage[dvipsnames]{xcolor} % for 'navy' color
\usepackage[a4paper,margin=25mm]{geometry} % set the margin for a4paper
% \usepackage{calligra}

% Math packages:
\usepackage{amsmath,amssymb}  % Enables advanced math environments like align, gather, and equation.
\usepackage{amsthm}  % Theorem infra
\usepackage{bbm}  % For indicator
\usepackage{amssymb}  % Provides additional math symbols (e.g., \mathbb, \leqslant).
\usepackage{mathrsfs} % For math script fonts

% Customize enumerate for no indentation
\usepackage{enumitem}

\usepackage{listings}
\usepackage{xcolor}
\usepackage{geometry}
\usepackage{subcaption}
\usepackage{wrapfig} % for wrapping text around figures

% indicator function
\newcommand{\ind}{\mathbbm{1}}

\geometry{
    top=0.3in,
    bottom=0.3in,
    left=0.5in,
    right=0.5in,
}

\lstset{
    language=R,
    basicstyle=\ttfamily\fontsize{8}{9}\selectfont,
    keywordstyle=\color{blue},
    commentstyle=\color{gray},
    stringstyle=\color{red},
    numbers=left,
    numberstyle=\tiny,
    frame=single,
    breaklines=true
}

\begin{document}
    {\centering
    \Large\textbf{Monte Carlo Methods Graded Lab 3}\\[0.5em]
    \normalsize Polina Ptukha, Xiaopeng Zhang\\[0.3em]
    December 11, 2025\\[1em] % or \today
    }

    \subsection*{Control Variates}
    The probability density function of the random variable \(X\) is given by
    \[f(x) = \frac{3}{4} (1-x^2) \ind_{[-1,1]}(x) \]
    To apply the control variates method, we first need to compute the even moments of \(X\):

    \begin{minipage}[t]{0.55\columnwidth}
        \vspace*{\fill}
        We can find a general formula for the even moments as follows:
        \begin{align*}
        \mathbb{E}(X^{2k}) 
        &= \int_{-\infty}^{+\infty} x^{2k} f(x)\,dx = \frac{3}{4} \int_{-1}^{1} x^{2k}(1-x^2)\,dx \\
        &= \frac{3}{4} \cdot 2 \int_0^1 x^{2k}(1-x^2)\,dx \\
        &= \frac{3}{2} \int_0^1 (x^{2k} - x^{2k+2})\,dx \\
        &= \frac{3}{2} \left[ \frac{1}{2k+1} - \frac{1}{2k+3} \right] = \frac{3}{(2k+1)(2k+3)}
        \end{align*}
    \end{minipage}\hfill
    \begin{minipage}[t]{0.40\columnwidth}
        \vspace*{\fill}

        Therefore, we have:
        
        \vspace*{1em}

        $\mathbb{E}(X^2) = \frac{1}{5},$

        \vspace*{1em}
    
        $\mathbb{E}(X^4) = \frac{3}{35},$
        
        \vspace*{1em}

        $\mathbb{E}(X^6) = \frac{1}{21}.$

        \vspace*{1em}

        $\mathbb{E}(X^8) = \frac{1}{33},$

        \vspace*{1em}

        $\mathbb{E}(X^{12}) = \frac{1}{65}.$

    \end{minipage}

    \vspace*{1em}
    
    Using \(h_{0,1} = x^4\) and \(h_{0,2} = x^6\) as control variables, we compute the optimal coefficients:
    \begin{align*}
        \beta_1^* &= \frac{\mathrm{Cov}(f(X), h_{0,1}(X))}{\mathrm{Var}(h_{0,1}(X))} = \frac{\mathbb{E}(X^2 X^4) - \mathbb{E}(X^2)\mathbb{E}(X^4)}{\mathbb{E}(X^8) - \left(\mathbb{E}(X^4)\right)^2} = \frac{\frac{1}{21} - \frac{1}{5}\cdot\frac{3}{35}}{\frac{1}{33} - \left(\frac{3}{35}\right)^2} = \frac{77}{58} \approx 1.33 \\
        \beta_2^* &= \frac{\mathrm{Cov}(f(X), h_{0,2}(X))}{\mathrm{Var}(h_{0,2}(X))} = \frac{\mathbb{E}(X^2 X^6) - \mathbb{E}(X^2)\mathbb{E}(X^6)}{\mathbb{E}(X^{12}) - \left(\mathbb{E}(X^6)\right)^2} = \frac{\frac{1}{33} - \frac{1}{5}\cdot\frac{1}{21}}{\frac{1}{65} - \left(\frac{1}{21}\right)^2} = \frac{819}{517} \approx 1.58.
    \end{align*}

    \subsection*{Comparison of estimators}
    The true value of the integral is equal to
    \[\mathbb{E}[X^2] = \frac{1}{5}\]
    Now let us compare values of the estimators with the true value from \(n = 10^5\) to \(n = 10^6\).
    
    \noindent
    \begin{minipage}[t]{0.65\columnwidth}
        \begin{center}
            \includegraphics[width=0.90\columnwidth]{imgs/comp.png}
            \label{fig:estimators}
        \end{center}
    \end{minipage}\hfill
    \begin{minipage}[c]{0.35\columnwidth}
        \vspace{-22em}
        The calculation with the highest \(n\) yields
        \[|\hat{I}^{MC}_n - \hat{I}^{CV1}_n| \approx 15 \times 10^{-4}\]
        \[|\hat{I}^{MC}_n - \hat{I}^{CV2}_n| \approx 2 \times 10^{-3}\]
        \[|\hat{I}^{CV1}_n - \hat{I}^{CV2}_n| \approx 5 \times 10^{-4}\]

        Let us compare efficiency of estimators.
    
        We will do it by estimating \(R\):
        \[R_{1,2}=\frac{cost1 * var1}{cost2 * var2}\]
        Thus we obtain following values
        \[R_{CV1,MC} = 0.1608956\] We can conclude that CV1 estimator is more efficient than MC as \(R<1\).
        \[R_{CV2,MC} = 0.3900529\] We can conclude that CV2 estimator is more efficient than MC as \(R<1\).
    \end{minipage}

\end{document}