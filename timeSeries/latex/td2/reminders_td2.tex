\documentclass[a4paper, 12pt]{article}

\setlength{\parindent}{0pt} % Disable the indent for new paragraphs

% Aesthetic packages
\usepackage[T1]{fontenc} % select T1 font encoding: suitable for Western European Latin scripts
\usepackage[most]{tcolorbox}  % for styled boxes
\usepackage[dvipsnames]{xcolor} % for 'navy' color
\usepackage[a4paper,margin=25mm]{geometry} % set the margin for a4paper
% \usepackage{calligra}

% Math packages:
\usepackage{amsmath,amssymb}  % Enables advanced math environments like align, gather, and equation.
\usepackage{amsthm}  % Theorem infra
\usepackage{bbm}  % For indicator
\usepackage{amssymb}  % Provides additional math symbols (e.g., \mathbb, \leqslant).
\usepackage{mathrsfs} % For math script fonts

% Customize enumerate for no indentation
\usepackage{enumitem}

% macro pour le titre
\newcommand{\demo}{\noindent\textbf{Demonstration: }}
\newcommand{\rpos}{\noindent\textbf{Réponse: }}

% indicator function
\newcommand{\ind}{\mathbbm{1}}

% Exercise Box
\newtcolorbox{exercisebox}[1]{
  width=\textwidth,
  colback=gray!10,
  colframe=black,
  boxrule=0.8pt,
  arc=4pt,
  fonttitle=\bfseries,
  title=Exercise #1,
  left=10pt, right=10pt,     % horizontal padding inside the box
  top=6pt, bottom=6pt        % vertical padding
}

% Define Answer environment
\newtcolorbox{answer}[1][]{
  breakable,
  enhanced,
  colback=Blue!5,         % no background color
  colframe=Blue,     % navy blue border
  boxrule=1.5pt,         % frame line thickness
  rounded corners,         % or 'rounded corners' if you prefer
  #1
}

% Answer enviroment with enumerate
\newenvironment{answerenum}{
    \begin{enumerate}[label=\textbf{\arabic*.}, leftmargin=*, labelindent=0pt, itemindent=0pt]
}{
    \end{enumerate}
}

% Remark Box
\newtcolorbox{remarkbox}{
  width=\textwidth,
  colback=yellow!10,    % light yellow background
  colframe=orange!80!black,
  boxrule=0.7pt,
  arc=3pt,
  fonttitle=\bfseries,
  title=Remark,
  left=8pt, right=8pt,
  top=5pt, bottom=5pt
}

\newcommand{\remark}[1]{%
  \begin{remarkbox}#1\end{remarkbox}%
}

% Define theorem-like environments
\theoremstyle{definition}
\newtheorem{definition}{Definition}[section]

\theoremstyle{plain}
\newtheorem{proposition}[definition]{Proposition}
\newtheorem{theorem}[definition]{Theorem}
    
% \theoremstyle{remark}
% \newtheorem{remark}[definition]{Remark}

\theoremstyle{definition}
\newtheorem{exercise}[definition]{Exercise}

\title{Reminders for TD2 -- Time Series}
\date{\today}
\author{ZXP}

\begin{document}

\maketitle

\section{Linear Processes and Autocovariance}

\subsection{Definition: Linear Process}
\textbf{Statement:} If $Z \sim \text{BB}(0,1)$ and $\alpha \in \ell^1(\mathbb{Z})$, then $X = F_\alpha(Z)$ is a stationary process with mean 0 and autocovariance function:
\[
\gamma_X(h) = \sum_{k \in \mathbb{Z}} \alpha_k \alpha_{k+h}
\]

\textit{Relevant for: Exercise 1}

\subsection{Summability and Decorrelation}
\textbf{Properties:}
\begin{itemize}
\item The autocovariance function $\gamma_X$ of a stationary process is bounded: $|\gamma_X(h)| \leq \gamma_X(0)$ for all $h \in \mathbb{Z}$.
\item For a linear process $X = F_\alpha(Z)$ with $\alpha \in \ell^1(\mathbb{Z})$, the autocovariance $\gamma_X(h) = \sum_{k \in \mathbb{Z}} \alpha_k \alpha_{k+h}$ is summable.
\item Cauchy-Schwarz inequality: $|\gamma_X(h)|^2 \leq \gamma_X(0)^2$ and $\sum_{h \in \mathbb{Z}} |\gamma_X(h)| \leq \left(\sum_{k \in \mathbb{Z}} |\alpha_k|\right)^2 < \infty$.
\end{itemize}

\textit{Relevant for: Exercise 1}

\subsection{Theorem: Properties of Autocovariance (Theorem 1.1)}
The autocovariance function $\gamma_X$ is always:
\begin{enumerate}
\item \textbf{Symmetric:} $\gamma_X(s,t) = \gamma_X(t,s)$ for all $s,t \in \mathbb{Z}$
\item \textbf{Positive definite:} For all $n \geq 1$, $(t_1,\ldots,t_n) \in \mathbb{Z}^n$, $(\lambda_1,\ldots,\lambda_n) \in \mathbb{R}^n$:
\[
\sum_{i=1}^n \sum_{j=1}^n \lambda_i\lambda_j\gamma_X(t_i,t_j) \geq 0
\]
\end{enumerate}

\textit{Relevant for: Exercise 1}

\subsection{Corollary: Autocovariance of Stationary Processes (Corollary 1.1)}
When $X$ is stationary, $\gamma_X: \mathbb{Z} \to \mathbb{R}$ is:
\begin{enumerate}
\item \textbf{Even:} $\gamma_X(h) = \gamma_X(-h)$ for all $h \in \mathbb{Z}$
\item \textbf{Positive definite:} $\sum_{i=1}^n \sum_{j=1}^n \lambda_i\lambda_j\gamma_X(t_i-t_j) \geq 0$
\item \textbf{Bounded:} $|\gamma_X(h)| \leq \gamma_X(0)$ for all $h \in \mathbb{Z}$
\end{enumerate}

\textit{Relevant for: Exercise 1}

\section{Filtering and $L^2$ Convergence}

\subsection{Theorem: Filtering of Bounded Processes (Theorem 2.1)}
Given $\alpha = (\alpha_k)_{k \in \mathbb{Z}} \in \ell^1(\mathbb{Z})$, $p \in [1,\infty[$, and process $X$ bounded in $L^p$:
\[
\sup_{t \in \mathbb{Z}} \|X_t\|_{L^p} < \infty
\]

Then:
\begin{enumerate}
\item For all $t \in \mathbb{Z}$, the sum $Y_t := \sum_{k \in \mathbb{Z}} \alpha_k X_{t-k}$ is a.s. absolutely convergent
\item Process $Y = (Y_t)_{t \in \mathbb{Z}}$ is bounded in $L^p$
\item For all $t \in \mathbb{Z}$:
\[
\sum_{k \in [-n,n]} \alpha_k X_{t-k} \xrightarrow[n \to \infty]{a.s., L^p} Y_t
\]
\end{enumerate}

\textit{Relevant for: Exercise 2}

\subsection{Weak Filtering with Bilateral Series}
\textbf{Key idea:} Even when $\alpha \notin \ell^1(\mathbb{Z})$, if $\alpha \in \ell^2(\mathbb{Z})$ and $Z$ is a centered white noise with unit variance, the series $\sum_{k \in \mathbb{Z}} \alpha_k Z_{t-k}$ can still converge in $L^2$.

\textbf{Method:} For fixed $t \in \mathbb{Z}$, consider the partial sums $S_n = \sum_{k=-n}^n \alpha_k Z_{t-k}$. Since $(Z_t)$ is a white noise:
\[
\mathbb{E}[S_n^2] = \sum_{k=-n}^n \alpha_k^2 \mathbb{E}[Z_{t-k}^2] = \sum_{k=-n}^n \alpha_k^2
\]

If $\alpha \in \ell^2(\mathbb{Z})$, then $(S_n)$ is Cauchy in $L^2$, hence converges to some $X_t \in L^2$.

\textit{Relevant for: Exercise 2}

\subsection{Stationarity via Filtering}
\textbf{Corollary 2.1:} If $\alpha \in \ell^1(\mathbb{Z})$ and $X$ is stationary second-order, then $Y = F_\alpha(X)$ is well-defined, second-order, stationary with:
\[
\mu_Y = \mu_X \sum_{k \in \mathbb{Z}} \alpha_k, \quad \gamma_Y(h) = \sum_{j \in \mathbb{Z}} \sum_{k \in \mathbb{Z}} \alpha_j\alpha_k\gamma_X(h+j-k)
\]

\textbf{Extension to $\ell^2$ case:} Even if $\alpha \in \ell^2(\mathbb{Z}) \setminus \ell^1(\mathbb{Z})$, the filtered process $(X_t)$ defined as the $L^2$ limit is still stationary because:
\begin{itemize}
\item $\mathbb{E}[X_t] = 0$ (by linearity of expectation and orthogonality)
\item $\gamma_X(t,s) = \text{Cov}(X_t, X_s)$ depends only on $|t-s|$ due to the stationarity of $(Z_t)$
\end{itemize}

\textit{Relevant for: Exercise 2}

\section{Autoregressive (AR) Processes}

\subsection{Theorem: AR(1) Processes (Theorem 3.1)}
Equation: $X_t = \phi X_{t-1} + Z_t$ where $Z \sim \text{BB}(0,1)$

\textbf{Case $|\phi| = 1$:} No stationary solution

\textbf{Case $|\phi| < 1$:} Unique stationary solution:
\[
X_t = \sum_{k=0}^\infty \phi^k Z_{t-k}
\]
\begin{itemize}
\item Linear, \textbf{causal} (depends only on past noise)
\item $\mu_X = 0$, $\gamma_X(h) = \frac{\phi^{|h|}}{1-\phi^2}$
\end{itemize}

\textbf{Case $|\phi| > 1$:} Unique stationary solution:
\[
X_t = -\sum_{k=1}^\infty \frac{Z_{t+k}}{\phi^k}
\]
\begin{itemize}
\item Linear, \textbf{non-causal} (depends on future noise)
\item $\mu_X = 0$, $\gamma_X(h) = \frac{\phi^{-|h|}}{\phi^2-1}$
\end{itemize}

\textit{Relevant for: Exercise 3}

\subsection{General AR(p) Processes and Stationarity}
For the general autoregressive equation:
\[
X_t = \sum_{i=1}^n a_i X_{t-i} + Z_t
\]

Define the \textbf{characteristic polynomial}:
\[
\Phi(z) = 1 - \sum_{i=1}^n a_i z^i
\]

\textbf{Criterion for stationary solution:} The equation admits a stationary solution if and only if $\Phi(z)$ has no roots of modulus 1.

\textbf{Key observation:} If $\sum_{i=1}^n a_i = 1$, then $\Phi(1) = 0$, so $z=1$ is a root. Similarly, if $\sum_{i=1}^n (-1)^i a_i = 1$, then $\Phi(-1) = 0$, so $z=-1$ is a root. Since both $|1| = 1$ and $|-1| = 1$, no stationary solution exists.

\textit{Relevant for: Exercise 3}

\subsection{Causality of AR Processes}
\textbf{Definition:} A process $X$ is \textbf{causal} if $X_t \in \overline{\text{Vect}}(Z_t, Z_{t-1}, Z_{t-2}, \ldots)$ for all $t \in \mathbb{Z}$ (closure in $L^2$).

\textbf{Criterion:} The stationary solution of $\Phi(B)X_t = Z_t$ is causal if and only if $\Phi(z)$ has no roots in the closed unit disk $\{z \in \mathbb{C}: |z| \leq 1\}$.

\textit{Relevant for: Exercise 3}

\section{Filter Invertibility}

\subsection{Power Series Representation}
For $\alpha \in \ell^1(\mathbb{Z})$, define:
\[
P_\alpha(z) := \sum_{k \in \mathbb{Z}} \alpha_k z^k
\]

This is continuous on the unit circle $\mathbb{U} = \{z \in \mathbb{C}: |z|=1\}$.

\textbf{Key properties (Proposition 2.2):}
\begin{itemize}
\item $P_{\alpha*\beta}(z) = P_\alpha(z) \times P_\beta(z)$
\item $P_e(z) = 1$ where $e = (\mathbb{1}_{k=0})_{k \in \mathbb{Z}}$
\item \textbf{Injectivity:} $P_\alpha = P_\beta \Rightarrow \alpha = \beta$
\end{itemize}

\textit{Relevant for: Exercise 4}

\subsection{Theorem: Inversion of Polynomial Filters (Theorem 2.3)}
Let $\alpha \in \ell^1(\mathbb{Z})$ be a polynomial filter. The following are equivalent:
\begin{enumerate}
\item Filter $\alpha$ is invertible (i.e., $\exists \beta \in \ell^1(\mathbb{Z})$ such that $\alpha * \beta = e$)
\item Polynomial $P_\alpha(z) = \sum_{k \in \mathbb{Z}} \alpha_k z^k$ has no roots of modulus 1
\item The rational fraction $1/P_\alpha$ is expandable as a power series on the unit circle: $\exists \beta \in \ell^1(\mathbb{Z})$ such that
\[
\forall z \in \mathbb{U}, \quad \frac{1}{P_\alpha(z)} = \sum_{k \in \mathbb{Z}} \beta_k z^k
\]
\end{enumerate}

When this holds: $\alpha^{-1} = \beta$

\textit{Relevant for: Exercise 4}

\subsection{Practical Inversion Formulas}
For polynomial filters, use partial fraction decomposition. If $P_\alpha(z) = (z-a)^k$ for some $a \in \mathbb{C} \setminus \{0\}$:

\textbf{If $|a| > 1$:}
\[
\frac{1}{(z-a)^k} = \frac{1}{(-a)^k}\sum_{n=0}^\infty \binom{n+k-1}{k-1}z^n
\]
(The inverse is causal: $\beta_k = 0$ for $k < 0$)

\textbf{If $|a| < 1$:}
\[
\frac{1}{(z-a)^k} = \sum_{n=0}^\infty \binom{n+k-1}{k-1}\frac{a^n}{z^{n+k}}
\]
(The inverse is anti-causal: $\beta_k = 0$ for $k > 0$)

\textbf{If $|a| = 1$:} The filter is not invertible in $\ell^1(\mathbb{Z})$.

\textit{Relevant for: Exercise 4}

\subsection{Examples for Exercise 4}
\textbf{Case 1:} $\alpha_0 = 2$, $\alpha_1 = -1$, $\alpha_k = 0$ otherwise.
\begin{itemize}
\item $P_\alpha(z) = 2 - z = -z(1 - 2/z)$
\item Root: $z = 2$ (modulus $|2| = 2 > 1$)
\item Invertible, inverse is causal
\end{itemize}

\textbf{Case 2:} $\alpha_0 = 1$, $\alpha_1 = 2$, $\alpha_k = 0$ otherwise.
\begin{itemize}
\item $P_\alpha(z) = 1 + 2z$
\item Root: $z = -1/2$ (modulus $|-1/2| = 1/2 < 1$)
\item Invertible, inverse is anti-causal
\end{itemize}

\textbf{Case 3:} $\alpha_0 = 1$, $\alpha_1 = -1$, $\alpha_k = 0$ otherwise.
\begin{itemize}
\item $P_\alpha(z) = 1 - z$
\item Root: $z = 1$ (modulus $|1| = 1$)
\item \textbf{Not invertible} in $\ell^1(\mathbb{Z})$
\end{itemize}

\textit{Relevant for: Exercise 4}

\section{Abstract Filtering Theory}

\subsection{Banach Spaces for Time Series}
\textbf{Definition:} $E$ is the space of processes $(X_t)$ bounded in $L^2$:
\[
\|X\|_E = \sup_{t \in \mathbb{Z}} \|X_t\|_2 < \infty
\]

$E$ with this norm is a Banach space.

\textbf{Backward shift operator:} $B \in L(E)$ defined by:
\[
BX = (X_{t-1})_{t \in \mathbb{Z}}
\]

\textit{Relevant for: Exercise 5}

\subsection{Theorem: $B$ is an Isometry}
The backward shift operator $B$ on $E$ satisfies:
\begin{enumerate}
\item $B$ is linear and bijective
\item $\|BX\|_E = \|X\|_E$ for all $X \in E$ (isometry property)
\item $B^{-1}$ is the forward shift: $(B^{-1}X)_t = X_{t+1}$
\end{enumerate}

\textbf{Consequence:} $\|B\|_{L(E)} = 1$ and $B^n$ is also an isometry for all $n \in \mathbb{Z}$.

\textit{Relevant for: Exercise 5}

\subsection{Convergence of Operator Series}
\textbf{Proposition:} If $\alpha \in \ell^1(\mathbb{Z})$, then the series $\sum_{n \in \mathbb{Z}} \alpha_n B^n$ converges in $L(E)$.

\textbf{Proof idea:} Since $\|B^n\|_{L(E)} = 1$ for all $n$:
\[
\left\|\sum_{n \in \mathbb{Z}} \alpha_n B^n\right\|_{L(E)} \leq \sum_{n \in \mathbb{Z}} |\alpha_n| \|B^n\|_{L(E)} = \|\alpha\|_1 < \infty
\]

We denote: $\phi(\alpha) := \sum_{n \in \mathbb{Z}} \alpha_n B^n \in L(E)$

\textit{Relevant for: Exercise 5}

\subsection{Morphism Property}
\textbf{Theorem:} The map $\phi: \ell^1(\mathbb{Z}) \to L(E)$ is an algebra homomorphism:
\[
\phi(\alpha * \beta) = \phi(\alpha) \circ \phi(\beta)
\]

\textbf{Proof sketch:} Expand both sides and use the fact that $B^m \circ B^n = B^{m+n}$.

\textbf{Consequence:} If $\alpha$ is invertible in $\ell^1(\mathbb{Z})$ (i.e., $\exists \beta \in \ell^1(\mathbb{Z})$ with $\alpha*\beta = e$), then $\phi(\alpha)$ is invertible in $L(E)$ with:
\[
\phi(\alpha)^{-1} = \phi(\beta) = \phi(\alpha^{-1})
\]

\textit{Relevant for: Exercise 5}

\subsection{Injectivity of $\phi$}
\textbf{Theorem:} The map $\phi: \ell^1(\mathbb{Z}) \to L(E)$ is injective.

\textbf{Proof strategy:} Consider a white noise $Z = (Z_t)_{t \in \mathbb{Z}}$ with $\mathbb{E}[Z_t] = 0$ and $\mathbb{E}[Z_t^2] = 1$. Then $Z \in E$.

For $\alpha \in \ell^1(\mathbb{Z})$, we have:
\[
\phi(\alpha)Z = \sum_{n \in \mathbb{Z}} \alpha_n B^n Z = \sum_{n \in \mathbb{Z}} \alpha_n (Z_{t-n})_{t \in \mathbb{Z}} = (Y_t)_{t \in \mathbb{Z}}
\]
where $Y_t = \sum_{n \in \mathbb{Z}} \alpha_n Z_{t-n}$.

\textbf{Key observation:} The map $\alpha \mapsto Y$ is injective because we can recover $\alpha$ from $Y$ via:
\[
\alpha_n = \mathbb{E}[Y_0 Z_{-n}]
\]
(using orthogonality of white noise).

Thus, if $\phi(\alpha) = 0$, then $Y = 0$, so $\alpha = 0$.

\textit{Relevant for: Exercise 5}

\end{document}