\setlength{\parindent}{0pt} % Disable the indent for new paragraphs

% Aesthetic packages
\usepackage[T1]{fontenc} % select T1 font encoding: suitable for Western European Latin scripts
\usepackage[most]{tcolorbox}  % for styled boxes
\usepackage[dvipsnames]{xcolor} % for 'navy' color
\usepackage[a4paper,margin=25mm]{geometry} % set the margin for a4paper
% \usepackage{calligra}

% Math packages:
\usepackage{amsmath,amssymb}  % Enables advanced math environments like align, gather, and equation.
\usepackage{amsthm}  % Theorem infra
\usepackage{bbm}  % For indicator
\usepackage{amssymb}  % Provides additional math symbols (e.g., \mathbb, \leqslant).
\usepackage{mathrsfs} % For math script fonts

% Customize enumerate for no indentation
\usepackage{enumitem}

% macro pour le titre
\newcommand{\demo}{\noindent\textbf{Demonstration: }}
\newcommand{\rpos}{\noindent\textbf{Réponse: }}

% indicator function
\newcommand{\ind}{\mathbbm{1}}

% Exercise Box
\newtcolorbox{exercisebox}[1]{
  width=\textwidth,
  colback=gray!10,
  colframe=black,
  boxrule=0.8pt,
  arc=4pt,
  fonttitle=\bfseries,
  title=Exercise #1,
  left=10pt, right=10pt,     % horizontal padding inside the box
  top=6pt, bottom=6pt        % vertical padding
}

% Define Answer environment
\newtcolorbox{answer}[1][]{
  breakable,
  enhanced,
  colback=Blue!5,         % no background color
  colframe=Blue,     % navy blue border
  boxrule=1.5pt,         % frame line thickness
  rounded corners,         % or 'rounded corners' if you prefer
  #1
}

% Answer enviroment with enumerate
\newenvironment{answerenum}{
    \begin{enumerate}[label=\textbf{\arabic*.}, leftmargin=*, labelindent=0pt, itemindent=0pt]
}{
    \end{enumerate}
}

% Remark Box
\newtcolorbox{remarkbox}{
  width=\textwidth,
  colback=yellow!10,    % light yellow background
  colframe=orange!80!black,
  boxrule=0.7pt,
  arc=3pt,
  fonttitle=\bfseries,
  title=Remark,
  left=8pt, right=8pt,
  top=5pt, bottom=5pt
}

\newcommand{\remark}[1]{%
  \begin{remarkbox}#1\end{remarkbox}%
}

% Define theorem-like environments
\theoremstyle{definition}
\newtheorem{definition}{Definition}[section]

\theoremstyle{plain}
\newtheorem{proposition}[definition]{Proposition}
\newtheorem{theorem}[definition]{Theorem}
    
% \theoremstyle{remark}
% \newtheorem{remark}[definition]{Remark}

\theoremstyle{definition}
\newtheorem{exercise}[definition]{Exercise}